\documentclass[aspectratio=169]{beamer}

% --- Preamble Settings ---

% Theme and Color
% 'Boadilla' is a clean, minimal theme.
\usetheme{Boadilla}
% You might want a slightly darker colour for maths
\usecolortheme{whale}


% Packages for Math and Text
\usepackage{amsmath}
\usepackage{amsfonts}
\usepackage{amssymb}
\usepackage{amsthm} % For theorem environments if you don't use beamer's built-in ones
\usepackage{graphicx} % For images
\usepackage{hyperref} % For links
\usepackage{subcaption}
\usepackage[backend=biber,style=authoryear]{biblatex}

\addbibresource{presentation/ref.bib} % link to .bib file

% --- USYD Specific Information ---
\title[Pure Maths Honours]{Navigating the Pachner Graph}
\subtitle{Algorithms for Searching and Sampling Triangulations}
\author{Daniel Bruwel}
\institute[USYD]{
    School of Mathematics and Statistics\\
    Faculty of Science\\
    The University of Sydney
}
\date{October 16, 2025}
\logo{\includegraphics[height=1cm]{presentation/usy_mb1_cmyk_stacked_logo.png}}

% --- Custom Environments (for Maths) ---
% \newtheorem{theorem}{Theorem}
% \newtheorem{lemma}[theorem]{Lemma}
% \newtheorem{definition}{Definition}
\newtheorem{proposition}[theorem]{Proposition}
% \newtheorem{corollary}[theorem]{Corollary}
\theoremstyle{remark}
\newtheorem{remark}{Remark}

% --- Start Document ---
\begin{document}

% -------------------------
% 1. Title Page
% -------------------------
\begin{frame}
    \titlepage
\end{frame}

% -------------------------
% 2. Table of Contents
% -------------------------
\begin{frame}{Outline}
    \tableofcontents
\end{frame}

% -------------------------
% 3. Background
% -------------------------
\section{Background}
\begin{frame}{Manifold}
    \begin{itemize}
        \item Locally looks like $\mathbb{R}^n$
        \item Different points can be distinguished
        \item The space is not unmanageably large
    \end{itemize}
\end{frame}
\begin{frame}{Manifold}
    \begin{Definition}[Locally Euclidean]
        A topological space $(X, \tau)$ is called ``locally Euclidean'' if there exists an integer $n>0$ such that every point has a neighbourhood that is homeomorphic to $\mathbb{R}^n$.
    \end{Definition}
    \begin{Definition}[Manifold]
        A topological space $(M, \tau)$ is called a ``manifold'' if it is Hausdorff, second countable, and locally Euclidean.
    \end{Definition}
\end{frame}
\begin{frame}{Manifold}
    \begin{figure}
        \centering
        \includegraphics[width=0.8\linewidth]{presentation/figures/example_manifolds.png}
        \caption{Examples of manifolds from \cite{deng_2020}}
        \label{fig:example-manifolds}
    \end{figure}
\end{frame}

\begin{frame}{Triangulation}
    \begin{Definition}[Triangulation]
        A triangulation of a manifold $M$ is a homeomorphism between a simplicial complex $|A|\to M$
    \end{Definition}
\end{frame}
\begin{frame}{Triangulation}
    \begin{figure}
        \centering
        \includegraphics[width=0.5\linewidth]{presentation/figures/example_triangulation.png}
        \caption{Example triangulation of $T^2$ from \cite{ag2gaeh_2014}}
        \label{fig:example-triangulation}
    \end{figure}
\end{frame}
\begin{frame}{Triangulation}
    We do not require that the simplices all have unique vertices.
\end{frame}

\begin{frame}{$S^3$}
We can understand $S^3$ as $\mathbb{R}^3$ with a point at infinity.
    \begin{figure}
        \centering
        \includegraphics[width=0.3\linewidth]{presentation/figures/s1_as_r1.png}
        \caption{Representation of $S^1\cong \mathbb{R}^1\cup\{\infty\}$ from \cite{davis_2007}}
        \label{fig:s1-as-r1}
    \end{figure}
\end{frame}

\begin{frame}{Knots}
    \begin{itemize}
        \item Take an infinitely long piece of string and knot it
        \item Because the string is infinite, consider it as being inside $S^3$
        \item The string extends out to the point at infinity in both directions so forms a copy of $S^1$
        \item Two knots are the same if we can move the string around to get them, without passing the string through itself.
    \end{itemize}
\end{frame}
\begin{frame}{Knots}
    \begin{Definition}[Knot]
        A knot $K$ is an embedding of $S^1$ in $S^3$. 
    \end{Definition}
    \begin{Definition}[Knot Equivalence]
        Two knots are equivalent if they can be connected via ambient isotopy. 
    \end{Definition}
\end{frame}
\begin{frame}{Knots}
    \begin{figure}
        \centering
        \begin{minipage}{0.8\textwidth}
        \begin{subfigure}[b]{0.45\linewidth}
        \includegraphics[width=\textwidth]{presentation/figures/unknot.png}
        \caption{The Unknot form \cite{belk_2009}}
        \label{fig:unkot}
        \end{subfigure}
        \hfill
        \begin{subfigure}[b]{0.45\linewidth}
        \includegraphics[width=\textwidth]{presentation/figures/trefoil.png}
        \caption{The Trefoil knot form \cite{belk_2010}}
        \label{fig:trefoil}
        \end{subfigure}
        \end{minipage}
    \end{figure}
\end{frame}

\begin{frame}{Triangulation-knot connection}
    \begin{itemize}
        \item Start with a 1-vertex ($v$) triangulation of $S^3$
        \item Consider the edges starting and ending at $v$, they are $S^1$
        \item The edges correspond to knots. 
    \end{itemize}
\end{frame}
\begin{frame}{Triangulation-knot connection}
    \begin{Theorem}[\cite{haken_1961}]
        Every knot in $S^3$ can be realised as an edge in a triangulation of $S^3$
    \end{Theorem}
    \begin{proof}
        \textbf{Idea:} Take $S^3\backslash \nu(K)$ where $\nu$ is a tubular neighbourhood. This space can be triangulated, Fill in the tubular neighbourhood with a solid torus, the knot now exists as a series of edges. Collapse the triangulation to one vertex. 
    \end{proof}
\end{frame}

\begin{frame}{Pachner Moves / Graph}
    \begin{figure}
        \centering
        \includegraphics[width=0.5\linewidth]{presentation/figures/2_2_pachner_move.PNG}
        \caption{(2-2) Pachner move for a triangulation of a 2-manifold.}
        \label{fig:placeholder}
    \end{figure}
\end{frame}
\begin{frame}{Pachner Moves / Graph}
    \begin{figure}
        \centering
        \includegraphics[width=0.5\linewidth]{presentation/figures/2_3_pachner_move.png}
        \caption{(2-3) Pachner move for a triangulation of a 3-manifold from \cite{bene_2015}}
        \label{fig:placeholder}
    \end{figure}
\end{frame}

% -------------------------
% 4. Problem
% -------------------------
\section{Problem}
\begin{frame}{Problem Idea}
    Find triangulations that have many complex knots in them.
\end{frame}

\begin{frame}{Objective Function}
    \begin{itemize}
        \item Need to define complexity of a knot. The polynomial $\Delta_K(t)$ is a good candidate.
        \item Use $|\det(\Delta_K(t))|$ as a numeric measure of complexity.
        \item Compute for all edges and average to get objective function.
    \end{itemize}
\end{frame}
\begin{frame}{Objective Function}
    \begin{Definition}[Alexander Polynomial]
        The Alexander Polynomial $\Delta_K(t)$ is the generator of the first elementary ideal of the action of the deck transformations on the first homology of the infinite cyclic cover of the knot complement.
    \end{Definition}
    $\Delta_K(t)$ is a knot invariant. 
    
    E.g. for the unknot $\Delta_K(t)=1$, for the trefoil $\Delta_K(t)=t^2-t+1$
\end{frame}
\begin{frame}{Objective Function}
    \begin{align}
        \mathcal{O}&:T_1(S^3)\to\mathbb{R} \\
        &: T\mapsto \frac{1}{|E(T)|}\sum_{e\in E(T)} |\det{(\Delta_e(t))}|
    \end{align}
    Viewing each edge $e\in E(T)$ as a knot in $S^3$.
\end{frame}

\begin{frame}{Problem Formulation}
    \centering
    Our goal is to find triangulations where $\mathcal{O}$ is large.
\end{frame}

% -------------------------
% 5. Methods and Results
% -------------------------
\section{Methods and Results}
\subsection{Classical}
\begin{frame}{MCMC}
    \begin{itemize}
        \item \textbf{Idea:} Take some triangulation, look at its neighbours, pick one at random.
        \item \textbf{Problem:} For a triangulation of size $n$, there are far more neighbours of size $n+1$ than of size $n-1$, so the size will grow rapidly.
        \item \textbf{Solution:} As $n$ gets larger, make the probability of going up less than that of going down.
    \end{itemize}
\end{frame}
\begin{frame}{MCMC}
    \begin{figure}
        \centering
        \includegraphics[width=0.5\linewidth]{presentation/figures/mcmc_histogram.pdf}
        \caption{Histogram of MCMC Samples for the score function.}
        \label{fig:mcmc-histogram}
    \end{figure}
\end{frame}
\begin{frame}{MCMC}
    \begin{figure}
        \centering
        \includegraphics[width=0.5\linewidth]{presentation/figures/mcmc_log_histogram.pdf}
        \caption{Log histogram of MCMC Samples for the score function with power law fit of $\alpha\approx9.97$.}
        \label{fig:mcmc-log-histogram}
    \end{figure}
\end{frame}

\begin{frame}{Greedy Search}
    \begin{itemize}
        \item \textbf{Idea:} Start at some small triangulation, look at all the neighbours above and pick the best one. Repeat.
    \end{itemize}
\end{frame}
\begin{frame}{Greedy Search}
    
\end{frame}
\begin{frame}{Greedy Search}
    
\end{frame}
\begin{frame}{Greedy Search}
    \begin{itemize}
        \item \textbf{Idea:} Start at some small triangulation, look at all the neighbours above and pick the best one. Repeat.
        \item \textbf{Problem:} Number of neighbours is large, can get stuck in local minima.
        \item \textbf{Solution:} Use simulated annealing. 
    \end{itemize}
\end{frame}

\begin{frame}{Simulated Annealing}
    \begin{itemize}
        \item \textbf{Idea:} Propose a single neighbour at random with MCMC, accept it if its score is better, if the score is worse accept with some probability that depends on how much worse it is.
    \end{itemize}
\end{frame}
\begin{frame}{Simulated Annealing}
    \begin{figure}
        \centering
        \includegraphics[width=0.5\linewidth]{presentation/figures/sim_annealing.pdf}
        \caption{Simulated annealing results}
        \label{fig:sim-annealing}
    \end{figure}
\end{frame}
\begin{frame}{Simulated Annealing}
    \begin{figure}
        \centering
        \includegraphics[width=0.5\linewidth]{presentation/figures/sim_annealing_hist.pdf}
        \caption{Simulated annealing results compare to baseline.}
        \label{fig:sim-annealing-hist}
    \end{figure}
\end{frame}
\begin{frame}{Simulated Annealing}
\begin{itemize}
    \item \textbf{Idea:} Propose a single neighbour at random with MCMC, accept it if its score is better, if the score is worse accept with some probability that depends on how much worse it is.
    \item \textbf{Problem:} Each move is local and random, the strategy doesn't fundamentally uncover any suture in the space.
    \item \textbf{Solution:} Vector Embedding
\end{itemize}
    
\end{frame}

\subsection{Machine Learning}
\begin{frame}{Gradient Descent}

\end{frame}

\section{Conclusion and Future Work}
\begin{frame}{Summary and Conclusion}
    
\end{frame}

\begin{frame}{Future Work}
    
\end{frame}

% -------------------------
% 6. Bibliography/Acknowledgements
% -------------------------
\section{References and Acknowledgements}

\begin{frame}[allowframebreaks]{References}
    \printbibliography
\end{frame}

\begin{frame}{Acknowledgements}
    \begin{center}
        \textbf{Thank You!}

        \vspace{1cm}
        
        \textbf{Questions?}
    \end{center}
\end{frame}

\end{document}